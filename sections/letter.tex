\section*{A Letter from our Founder}

\footnotesize

``Large-scale automated transaction systems are imminent. As the initial choice for their architecture gathers economic and social momentum, it becomes increasingly difficult to reverse. Whichever approach prevails, it will likely have a profound and enduring impact on economic freedom, democracy, and our informational rights.''

``Restrictions on economic freedom may be furthered under the current approach. Markets are often manipulable by parties with special access to information about other participants' transactions. Information service providers and other major interests, for example, could retain control over various information and media distribution channels while synergistically consolidating their position with sophisticated marketing techniques that rely on gathering far-reaching information about consumers. Computerization has already allowed these and other organizations to grow to unprecedented size and influence; if computerization is continued along current lines, such domination might be further increased. But the computerization of information gathering and dissemination need not lead to centralization: integrating the payment system presented here with communication systems can give individuals and small organizations equal and unrestricted access to information distribution channels. Moreover, when information about the transactions of individuals and organizations is partitioned into separate, unlinkable relationships, the trend toward large-scale gathering of such information, with its potential for manipulation and domination of markets, can be reversed.''

``Attempts to computerize under the current approach threaten democracy as well. They are, as mentioned, likely to engender widespread opposition; the resulting stalemate would yield security mechanisms incapable of providing adequate prior restraint, thus requiring heavy surveillance, based on record-linking, for security. This surveillance might significantly chill individual participation and expression in group and public life. The inadequate security and the accumulation of personally identifiable records, moreover, pose national vulnerabilities. Additionally, the same sophisticated data acquisition and analysis techniques used in marketing are being applied to manipulating public opinion and elections as well. The opportunity exists, however, not only to reverse all these trends, by providing acceptable security without increased surveillance, but also to strengthen democracy. Voting, polling, and surveys, for example, could be conveniently conducted via the new systems; respondents could show relevant credentials pseudonymously, and centralized coordination would not be needed.''

``The new approach provides a practical basis for two new informational human rights that is unobtainable under the current approach. One is the right of individuals to parity with organizations in transaction system use. This is established in practice by individuals' parity in protecting themselves against abuses, resolving disputes, conferring proxy, and offering services. The other is the right of individuals to disclose only the minimum information necessary: in accessing information sources and distribution channels, in transactions with organizations, and---more fundamentally---in all the interactions that comprise an individual's informational life.''

``Advances in information technology have always been accompanied by major changes in society: the transition from tribal to larger hierarchical forms, for example, was accompanied by written language, and printing technology helped to foster the emergence of large-scale democracies. Coupling computers with telecommunications creates what has been called the ultimate medium---it is certainly a big step up from paper. One might then ask: To what forms of society could this new technology lead? The two approaches appear to hold quite different answers.''


\begin{flushright}
    --- David Chaum \\ 1987~\cite{Chaum85}
\end{flushright}

\normalsize